\documentclass[11pt, a4paper]{article}

% --- UNIVERSAL PREAMBLE BLOCK ---
\usepackage[a4paper, top=2.5cm, bottom=2.5cm, left=2cm, right=2cm]{geometry}
\usepackage{fontspec}

% Set main language to Chinese using xeCJK
\usepackage{xeCJK}
\setCJKmainfont{SimSun} % Standard Windows font for Chinese
\setmainfont{Times New Roman} % Standard English font


% Add because main language is not English
\usepackage{enumitem}
\setlist[itemize]{label=$\bullet$} 
% --------------------------------

\title{\textbf{小学二年级语文学习纲要:重点句式与语法}}
\author{语文教研组}
\date{}

\begin{document}

\maketitle

\section*{学习目标}
\begin{enumerate}
    \item \textbf{词法基础}:掌握“的、地、得”用法,学会准确使用量词及积累近反义词。
    \item \textbf{写完整句}:能够清楚地表达“谁/什么 + 干什么/怎么样”。
    \item \textbf{句式丰富}:学会使用关联词(因为……所以、如果……就……等)连接句子。
    \item \textbf{修辞启蒙}:初步掌握比喻、拟人等修辞手法,使句子更生动。
    \item \textbf{标点正确}:正确使用逗号、句号、问号、感叹号。
    \item \textbf{识字规律}:掌握“加偏旁”识字法,区分形近字。
    \item \textbf{句法入门}:初步了解句子的主要成分(主语、谓语、宾语)。
\end{enumerate}

\hrule
\vspace{0.5cm}

\section{词语积累与运用(词法基础)}
二年级是词汇积累的关键期,要掌握“的、地、得”的区别,学会准确使用量词,并积累常见的近义词和反义词。

\subsection{“的、地、得”的用法}
这三个字读音相同,但用法不同,请记住下面的口诀:

\begin{itemize}
    \item \textbf{名词前面“白勺”的}:形容词 + 的 + 名词 \\
    例:\textbf{美丽的}花朵、\textbf{清清的}河水
    \item \textbf{动词前面“土也”地}:形容词 + 地 + 动词 \\
    例:\textbf{飞快地}跑、\textbf{大声地}唱
    \item \textbf{动词后面“双人”得}:动词 + 得 + 形容词 \\
    例:跑\textbf{得快}、跳\textbf{得高}
\end{itemize}

\textbf{【练习】选词填空(的、地、得)}
\begin{enumerate}
    \item 蓝蓝( \hspace{1cm} )天空
    \item 慢慢( \hspace{1cm} )走
    \item 玩( \hspace{1cm} )真开心
    \item 可爱( \hspace{1cm} )小猫在草地( \hspace{1cm} )打滚。
\end{enumerate}

\subsection{量词的用法}
量词是表示人、事物或动作单位的词,要搭配准确。

\begin{itemize}
    \item \textbf{动物类}:一\textbf{头}牛、一\textbf{匹}马、一\textbf{只}鸡、一\textbf{条}鱼
    \item \textbf{物品类}:一\textbf{支}笔、一\textbf{本}书、一\textbf{把}尺、一\textbf{张}纸、一\textbf{块}橡皮
    \item \textbf{植物类}:一\textbf{棵}树、一\textbf{朵}花、一\textbf{片}叶子
\end{itemize}

\textbf{【练习】填上合适的量词}
\begin{enumerate}
    \item 一( \hspace{1cm} )大象
    \item 一( \hspace{1cm} )雨伞
    \item 一( \hspace{1cm} )日记
    \item 一( \hspace{1cm} )小河
\end{enumerate}

\subsection{近义词与反义词的积累}
\begin{itemize}
    \item \textbf{近义词}:意思相近的词。
    \begin{itemize}
        \item 漂亮——美丽 \quad 连忙——赶忙 \quad 假如——如果 \quad 十分——非常
    \end{itemize}
    \item \textbf{反义词}:意思相反的词。
    \begin{itemize}
        \item 宽——窄 \quad 忙——闲 \quad 升——降 \quad 寒冷——温暖 \quad 成功——失败
    \end{itemize}
\end{itemize}

\textbf{【练习】写一写}
\begin{enumerate}
    \item \textbf{写出近义词}:
    \begin{itemize}
        \item 中心——( \hspace{2cm} )
        \item 著名——( \hspace{2cm} )
    \end{itemize}
    \item \textbf{写出反义词}:
    \begin{itemize}
        \item 困难——( \hspace{2cm} )
        \item 危险——( \hspace{2cm} )
    \end{itemize}
\end{enumerate}

\newpage

\section{常用关联词与句式}
二年级重点掌握的关联词主要用于表达\textbf{并列}、\textbf{因果}、\textbf{转折}和\textbf{连续}动作。

\subsection{并列关系:一边……一边……}
\textbf{含义}:表示两个动作同时进行。必须是同一个人(或物体)可以同时做到的动作。

\begin{itemize}
    \item \textbf{例句}:
    \begin{itemize}
        \item 妈妈\textbf{一边}洗衣服,\textbf{一边}哼着歌。
        \item 不管是刮风还是下雨,他都坚持\textbf{一边}走路,\textbf{一边}背古诗。
    \end{itemize}
    \item \textbf{练习}:
    \begin{enumerate}
        \item 老师\underline{\hspace{2cm}}讲课,\underline{\hspace{2cm}}在黑板上写字。
        \item 我\underline{\hspace{2cm}}看电视,\underline{\hspace{2cm}}吃零食。
        \item 弟弟\underline{\hspace{2cm}}跑步,\underline{\hspace{2cm}}听音乐。
        \item 奶奶\underline{\hspace{2cm}}织毛衣,\underline{\hspace{2cm}}和我们聊天。
    \end{enumerate}
\end{itemize}

\subsection{因果关系:因为……所以……}
\textbf{含义}:前面说明原因,后面说明结果。也可以倒过来说“之所以……是因为……”来强调原因。

\begin{itemize}
    \item \textbf{例句}:
    \begin{itemize}
        \item \textbf{因为}今天下雨了,\textbf{所以}我们可以不用做课间操。
        \item 他\textbf{之所以}今天没来,\textbf{是因为}生病了。(倒装强调)
    \end{itemize}
    \item \textbf{练习}:
    \begin{enumerate}
        \item \underline{\hspace{2cm}}他学习很刻苦,\underline{\hspace{2cm}}这次考了100分。
        \item (倒装句)我\underline{\hspace{2cm}}喜欢这本书,\underline{\hspace{2cm}}里面的故事很有趣。
        \item \underline{\hspace{2cm}}天黑了,\underline{\hspace{2cm}}我们要回家了。
        \item 树叶\underline{\hspace{2cm}}变黄了,\underline{\hspace{2cm}}秋天到了。
    \end{enumerate}
\end{itemize}

\subsection{转折关系:虽然……但是……}
\textbf{含义}:前面的意思和后面的意思相反或相对。

\begin{itemize}
    \item \textbf{例句}:
    \begin{itemize}
        \item \textbf{虽然}天气很冷,\textbf{但是}同学们还是坚持锻炼身体。
        \item \textbf{虽然}这个苹果很小,\textbf{但是}它非常甜。
    \end{itemize}
    \item \textbf{练习}:
    \begin{enumerate}
        \item \underline{\hspace{2cm}}这道题很难,\underline{\hspace{2cm}}我还是做出来了。
        \item 爷爷\underline{\hspace{2cm}}年纪大了,\underline{\hspace{2cm}}身体依然很硬朗。
        \item \underline{\hspace{2cm}}雨下得很大,\underline{\hspace{2cm}}同学们还是准时到校了。
        \item \underline{\hspace{2cm}}我很想去玩,\underline{\hspace{2cm}}作业还没写完。
    \end{enumerate}
\end{itemize}

\subsection{连续动作:一……就……}
\textbf{含义}:表示一个动作发生后,紧接着发生另一个动作。

\begin{itemize}
    \item \textbf{例句}:
    \begin{itemize}
        \item 我\textbf{一}回到家,\textbf{就}开始写作业。
        \item 太阳\textbf{一}出来,雪\textbf{就}化了。
    \end{itemize}
    \item \textbf{练习}:
    \begin{enumerate}
        \item 铃声\underline{\hspace{1.5cm}}响,同学们\underline{\hspace{1.5cm}}冲出了教室。
        \item 我\underline{\hspace{1.5cm}}放学,\underline{\hspace{1.5cm}}回家帮妈妈做家务。
        \item 妈妈\underline{\hspace{1.5cm}}看到我考了满分,\underline{\hspace{1.5cm}}高兴地笑了。
        \item 春风\underline{\hspace{1.5cm}}吹,花儿\underline{\hspace{1.5cm}}开了。
    \end{enumerate}
\end{itemize}

\subsection{群体列举:有的……有的……还有的……}
\textbf{含义}:用来描写一群人或事物中不同的姿态或活动。

\begin{itemize}
    \item \textbf{例句}:
    \begin{itemize}
        \item 下课了,操场上真热闹,同学们\textbf{有的}跳绳,\textbf{有的}踢毽子,\textbf{还有的}在玩老鹰捉小鸡。
        \item 公园里的花开了,\textbf{有的}是红的,\textbf{有的}是黄的,\textbf{还有的}是紫的,五颜六色真好看。
    \end{itemize}
    \item \textbf{练习}:
    \begin{enumerate}
        \item 天上的白云形态各异,\underline{\hspace{1.5cm}}像小羊,\underline{\hspace{1.5cm}}像棉花糖,\underline{\hspace{1.5cm}}像奔跑的骏马。
        \item 体育课上,同学们\underline{\hspace{1.5cm}}打篮球,\underline{\hspace{1.5cm}}踢足球,\underline{\hspace{1.5cm}}在跑步。
        \item 果园里,\underline{\hspace{1.5cm}}是苹果,\underline{\hspace{1.5cm}}是梨,\underline{\hspace{1.5cm}}是葡萄。
        \item 这里的书真多,\underline{\hspace{1.5cm}}是童话书,\underline{\hspace{1.5cm}}是故事书,\underline{\hspace{1.5cm}}是漫画书。
    \end{enumerate}
\end{itemize}

\subsection{假设关系:如果……就……}
\textbf{含义}:表示一种设想,如果前面的情况发生了,就会有后面的结果。
\begin{itemize}
    \item \textbf{例句}:\textbf{如果}明天不下雨,我们\textbf{就}去公园玩。(也可以说“要是……就……”)
    \item \textbf{练习}:
    \begin{enumerate}
        \item \underline{\hspace{2cm}}我也有一双翅膀,\underline{\hspace{2cm}}能飞上蓝天了。
        \item \underline{\hspace{2cm}}你不快点走,我们\underline{\hspace{2cm}}要迟到了。
        \item \underline{\hspace{2cm}}我也能像孙悟空一样七十二变,那\underline{\hspace{2cm}}太好了。
        \item \underline{\hspace{2cm}}这周末天气好,爸爸\underline{\hspace{2cm}}带我去钓鱼。
    \end{enumerate}
\end{itemize}

\subsection{条件关系:只要……就……}
\textbf{含义}:强调付出努力(条件),就能得到结果。
\begin{itemize}
    \item \textbf{例句}:\textbf{只要}肯动脑筋,\textbf{就}能想出好办法。
    \item \textbf{练习}:
    \begin{enumerate}
        \item \underline{\hspace{2cm}}我们团结起来,\underline{\hspace{2cm}}能战胜困难。
        \item \underline{\hspace{2cm}}我们坚持到底,\underline{\hspace{2cm}}一定能成功。
        \item \underline{\hspace{2cm}}你认真听讲,\underline{\hspace{2cm}}能听懂老师的话。
        \item \underline{\hspace{2cm}}平时多积累,写作文\underline{\hspace{2cm}}不难了。
    \end{enumerate}
\end{itemize}

\subsection{顺序关系:先……再……}
\textbf{含义}:表示事情的先后顺序,写日记常用。
\begin{itemize}
    \item \textbf{例句}:放学回家,我\textbf{先}写作业,\textbf{再}看电视。
    \item \textbf{练习}:
    \begin{enumerate}
        \item 早上起床,我\underline{\hspace{2cm}}刷牙,\underline{\hspace{2cm}}洗脸。
        \item 做手工时,我们要\underline{\hspace{2cm}}剪纸,\underline{\hspace{2cm}}涂颜色。
        \item 吃饭前,要\underline{\hspace{2cm}}洗手,\underline{\hspace{2cm}}吃饭。
        \item 我\underline{\hspace{2cm}}把作业写完,\underline{\hspace{2cm}}出去玩。
    \end{enumerate}
\end{itemize}

\subsection{递进关系:不但……而且……}
\textbf{含义}:后面的意思比前面的更进一层,常用来夸奖别人。
\begin{itemize}
    \item \textbf{例句}:小明\textbf{不但}会画画,\textbf{而且}画得很好。
    \item \textbf{练习}:
    \begin{enumerate}
        \item 这里的风景\underline{\hspace{2cm}}美,\underline{\hspace{2cm}}空气也很新鲜。
        \item 他\underline{\hspace{2cm}}聪明,\underline{\hspace{2cm}}很勤奋。
        \item 这座桥\underline{\hspace{2cm}}坚固,\underline{\hspace{2cm}}美观。
        \item 苹果\underline{\hspace{2cm}}好看,\underline{\hspace{2cm}}好吃。
    \end{enumerate}
\end{itemize}

\subsection{总复习:选词填空(综合挑战)}

\begin{enumerate}
    \item ( \hspace{2cm} )我们是中国人,( \hspace{2cm} )我们要爱自己的祖国。
    \item 爷爷( \hspace{2cm} )看报纸,( \hspace{2cm} )听广播。
    \item ( \hspace{2cm} )明天不下雨,我们( \hspace{2cm} )去春游。
    \item ( \hspace{2cm} )我知道这道题怎么做,( \hspace{2cm} )我没时间写了。
    \item ( \hspace{2cm} )上课铃响了,同学们( \hspace{2cm} )安静了下来。
    \item ( \hspace{2cm} )我们努力学习,( \hspace{2cm} )能取得好成绩。
\end{enumerate}

\newpage

\section{基础修辞手法}
二年级主要要求掌握\textbf{比喻}和\textbf{拟人},这是看图写话加分的关键。

\subsection{比喻句(打比方)}
\textbf{公式}:本体 + 像 + 喻体

\begin{itemize}
    \item \textbf{例句}:
    \begin{itemize}
        \item 弯弯的\textbf{月亮}像一只小小的\textbf{船}。
        \item 平静的\textbf{湖面}像一面巨大的\textbf{镜子}。
    \end{itemize}
    \item \textbf{注意}:本体和喻体必须是两种不同的事物,但要有相似点。
    \item \textbf{练习}:
    \begin{enumerate}
        \item 细细的春雨像\underline{\hspace{5cm}}。
        \item \underline{\hspace{5cm}}像一个大火球。
    \end{enumerate}
\end{itemize}

\subsection{拟人句(把物当人写)}
\textbf{含义}:赋予植物、动物或物体人的动作、表情或语言。

\begin{itemize}
    \item \textbf{例句}:
    \begin{itemize}
        \item 小鸟在枝头\textbf{唱歌}。
        \item 花儿在春风中\textbf{张开了笑脸}。
    \end{itemize}
    \item \textbf{练习}:
    \begin{enumerate}
        \item 蝴蝶在花丛中\underline{\hspace{5cm}}。
        \item 柳树\underline{\hspace{5cm}}。
    \end{enumerate}
\end{itemize}

\section{句型转换(难点)}

\subsection{“把”字句与“被”字句互换}
\begin{itemize}
    \item \textbf{公式}:
    \begin{itemize}
        \item 把字句:主动者 + 把 + 被动者 + 动作
        \item 被字句:被动者 + 被 + 主动者 + 动作
    \end{itemize}
    \item \textbf{例句}:
    \begin{itemize}
        \item 原句:大水冲走了龙王庙。
        \item 把字句:大水\textbf{把}龙王庙冲走了。
        \item 被字句:龙王庙\textbf{被}大水冲走了。
    \end{itemize}
    \item \textbf{练习}:请将下列句子分别改写为“把”字句和“被”字句。
    \begin{enumerate}
        \item 我洗干净了红领巾。
        \begin{itemize}
            \item 把字句:\underline{\hspace{8cm}}
            \item 被字句:\underline{\hspace{8cm}}
        \end{itemize}
        \item 大风吹走了树叶。
        \begin{itemize}
            \item 把字句:\underline{\hspace{8cm}}
            \item 被字句:\underline{\hspace{8cm}}
        \end{itemize}
        \item 妈妈打扫了房间。
        \begin{itemize}
            \item 把字句:\underline{\hspace{8cm}}
            \item 被字句:\underline{\hspace{8cm}}
        \end{itemize}
        \item 熊猫吃掉了竹子。
        \begin{itemize}
            \item 把字句:\underline{\hspace{8cm}}
            \item 被字句:\underline{\hspace{8cm}}
        \end{itemize}
    \end{enumerate}
\end{itemize}

\subsection{把句子写具体(扩句)}
\textbf{含义}:在句子中加入形容词(……的……、……地……)。
\begin{itemize}
    \item \textbf{例句}:
    \begin{itemize}
        \item 原句:小鱼游来游去。
        \item 扩句:\textbf{可爱的}小鱼在\textbf{清澈的}水里\textbf{快活地}游来游去。
    \end{itemize}
    \item \textbf{练习}:请把下面的句子写具体(至少加两个修饰词)。
    \begin{enumerate}
        \item 同学们在读书。
        \begin{itemize}
            \item 扩句:\underline{\hspace{10cm}}
        \end{itemize}
        \item 小鸟在唱歌。
        \begin{itemize}
            \item 扩句:\underline{\hspace{10cm}}
        \end{itemize}
        \item 汽车在马路上跑。
        \begin{itemize}
            \item 扩句:\underline{\hspace{10cm}}
        \end{itemize}
        \item 星星在眨眼。
        \begin{itemize}
            \item 扩句:\underline{\hspace{10cm}}
        \end{itemize}
    \end{enumerate}
\end{itemize}

\section{标点符号的使用}
二年级要求能根据语气正确使用标点,并初步掌握人物对话的标点用法。

\begin{enumerate}
    \item \textbf{逗号(,)}:表示句子中间的停顿。(例:大树长出了绿叶,草地变绿了。)
    \item \textbf{句号(。)}:话说完了,语气平缓。(例:今天天气真好。)
    \item \textbf{问号(?)}:有疑问,或者发问。(例:你的作业写完了吗?)
    \item \textbf{感叹号(!)}:语气强烈,表示开心、惊讶、生气等。(例:快点走!)
    \item \textbf{冒号和引号( : “” )}:用在“说”、“问”等词后面,引用人物说的话。
    \begin{itemize}
        \item \textbf{口诀}:提示语在前面,冒号引号紧相连。
        \item \textbf{例句}:老师\textbf{说:“}同学们,下课了。\textbf{”}
    \end{itemize}
\end{enumerate}

\subsection{练一练:小小标点指挥官}
\textbf{请在下面的( )里填上合适的标点符号。}

\vspace{0.3cm}

\textbf{1. 基础标点练习(注意逗号和句号的区别):}
\begin{enumerate}
    \item 这个西瓜真甜啊( \hspace{0.5cm} )
    \item 你知道我想去哪里吗( \hspace{0.5cm} )
    \item 春天来了( \hspace{0.5cm} )小草从土里探出头来( \hspace{0.5cm} )
    \item 我喜欢吃苹果( \hspace{0.5cm} )也喜欢吃香蕉( \hspace{0.5cm} )
    \item 这里的风景真美( \hspace{0.5cm} )空气也很新鲜( \hspace{0.5cm} )
\end{enumerate}

\vspace{0.3cm}

\textbf{2. 趣味对话(重点练习冒号和引号):}
\begin{quote}
    \setlength{\baselineskip}{1.8em} % 增加行间距方便填写
    \qquad 一天,小猫在河边遇到了小鸭子。
    
    \qquad 小猫好奇地问( \hspace{0.5cm} )( \hspace{0.5cm} )小鸭子,河里的水冷不冷呀( \hspace{0.5cm} )( \hspace{0.5cm} )
    
    \qquad 小鸭子笑着说( \hspace{0.5cm} )( \hspace{0.5cm} )水里很暖和,快下来一起玩吧( \hspace{0.5cm} )( \hspace{0.5cm} )
\end{quote}

\newpage

\section{加偏旁变新字(字族识字)}
掌握“加偏旁”识字法,重点在于\textbf{根据偏旁判断字义}。

\subsection{“巴”字家族}
\begin{itemize}
    \item \textbf{父 + 巴 = 爸}(爸爸)
    \item \textbf{扌 + 巴 = 把}(把手、把握)
    \item \textbf{口 + 巴 = 吧}(好吧、走吧)
    \item \textbf{爪 + 巴 = 爬}(爬山、爬行)
\end{itemize}
\textbf{例句}:\textbf{爸}爸\textbf{把}门关上了。我们一起去\textbf{爬}山\textbf{吧}!

\vspace{0.3cm}
\textbf{【练习】选字填空(把、爸、吧、爬)}
\begin{enumerate}
    \item 你( \hspace{1cm} )作业写完了吗?
    \item 宝宝在地上( \hspace{1cm} )来( \hspace{1cm} )去。
\end{enumerate}

\subsection{“青”字家族(重点)}
\begin{itemize}
    \item \textbf{氵 + 青 = 清}(清水)$\rightarrow$ 与水有关
    \item \textbf{日 + 青 = 晴}(晴天)$\rightarrow$ 与太阳有关
    \item \textbf{目 + 青 = 睛}(眼睛)$\rightarrow$ 与眼睛有关
    \item \textbf{讠 + 青 = 请}(请客)$\rightarrow$ 与说话有关
    \item \textbf{忄 + 青 = 情}(心情)$\rightarrow$ 与心情有关
\end{itemize}
\textbf{顺口溜}:有水方说\textbf{清},有日天气\textbf{晴},有目是眼\textbf{睛},有言去邀\textbf{请},有心\textbf{情}意浓。

\vspace{0.3cm}
\textbf{【练习】选字填空(清、晴、睛、请、情)}
\begin{enumerate}
    \item 今天天气( \hspace{1cm} )朗,河水( \hspace{1cm} )可见底。
    \item ( \hspace{1cm} )问,你的眼( \hspace{1cm} )好了吗?
\end{enumerate}

\subsection{“包”字家族}
\begin{itemize}
    \item \textbf{饣 + 包 = 饱}(吃饱)$\rightarrow$ 与食物有关
    \item \textbf{⻊ + 包 = 跑}(跑步)$\rightarrow$ 与脚有关
    \item \textbf{扌 + 包 = 抱}(拥抱)$\rightarrow$ 与手有关
    \item \textbf{氵 + 包 = 泡}(气泡)$\rightarrow$ 与水有关
    \item \textbf{火 + 包 = 炮}(鞭炮)$\rightarrow$ 与火有关
\end{itemize}
\textbf{例句}:吃\textbf{饱}了才有力气\textbf{跑}步。妹妹\textbf{抱}着一个大\textbf{泡}泡机去放鞭\textbf{炮}。

\vspace{0.3cm}
\textbf{【练习】选字填空(抱、跑、饱、泡)}
\begin{enumerate}
    \item 小鱼在水里吐( \hspace{1cm} )( \hspace{1cm} )( \hspace{1cm} )( \hspace{1cm} )。
    \item 他吃得太( \hspace{1cm} )了,( \hspace{1cm} )不动了。
\end{enumerate}

\newpage

\section{句子的小火车:主语、谓语、宾语(拓展)}
一个完整的句子就像一列小火车,由\textbf{车头(谁)}、\textbf{车身(干什么)}和\textbf{车尾(对象)}组成。

\subsection{什么是主语、谓语、宾语?}
\begin{itemize}
    \item \textbf{主语(谁/什么)}:动作的发起者,是句子的主角。
    \item \textbf{谓语(干什么/怎么样)}:主语做的动作或状态。
    \item \textbf{宾语(对象)}:动作承受的对象。
\end{itemize}

\subsection{基础句式分析}
\textbf{例句1:我吃苹果。}
\begin{itemize}
    \item \textbf{我}(主语:谁?)
    \item \textbf{吃}(谓语:干什么?)
    \item \textbf{苹果}(宾语:吃的对象是什么?)
\end{itemize}

\textbf{例句2:小猫抓老鼠。}
\begin{itemize}
    \item \textbf{小猫}(主语) + \textbf{抓}(谓语) + \textbf{老鼠}(宾语)
\end{itemize}

\textbf{例句3:爸爸看报纸。}
\begin{itemize}
    \item \textbf{爸爸}(主语) + \textbf{看}(谓语) + \textbf{报纸}(宾语)
\end{itemize}

\subsection{练一练:找出句子里的“三兄弟”}
请在下面的句子中,用“\underline{\hspace{1cm}}”标出主语,用“$\sim\sim\sim$”标出谓语,用“( )”圈出宾语。

\vspace{0.3cm}
\begin{enumerate}
    \item 妈妈 做 晚饭。
    \item 弟弟 喝 牛奶。
    \item 哥哥 骑 自行车。
    \item 同学们 踢 足球。
\end{enumerate}

\subsection{特殊情况1:它们去哪了?(省略句)}
在对话或命令中,为了方便,有些成分会被省略。

\textbf{1. 省略主语(谁)}
\begin{itemize}
    \item \textbf{(你)请坐!} $\rightarrow$ 命令或请求时,常省略“你”。
    \item \textbf{(我们)下课了!} $\rightarrow$ 大家都知道指的是“我们”。
\end{itemize}

\textbf{2. 省略谓语或宾语}
\begin{itemize}
    \item 问:谁去擦黑板? $\rightarrow$ 答:\textbf{我}。(省略了“去擦黑板”)
    \item 问:你在干什么? $\rightarrow$ 答:\textbf{看书}。(省略了“我在”)
\end{itemize}

\subsection{特殊情况2:有些句子没有宾语}
有些动作不需要对象,所以句子只有主语和谓语,这也是完整的句子。
\begin{itemize}
    \item \textbf{太阳}(主语) \textbf{升起来了}(谓语)。
    \item \textbf{小鸟}(主语) \textbf{飞走了}(谓语)。
\end{itemize}

\subsection{问句里的秘密(疑问句分析)}
在问句里,主语、谓语、宾语的位置有时会变化,或者由疑问词(谁、什么、哪里)来担任。

\begin{itemize}
    \item \textbf{谁}(主语) 在唱歌(谓语)?
    \item \textbf{你}(主语) 吃(谓语) \textbf{什么}(宾语)?
    \item \textbf{爸爸}(主语) 去(谓语) \textbf{哪里}(宾语) 了?
\end{itemize}

\end{document}