\documentclass[12pt, a4paper]{article}

% --- PREAMBLE ---
\usepackage[a4paper, top=2cm, bottom=2cm, left=2cm, right=2cm]{geometry}
\usepackage{fontspec}
\usepackage{xcolor}
\usepackage{longtable} % 用于跨页长表格
\usepackage{booktabs}  % 用于美观的表格线
\usepackage{array}     % 用于表格列宽控制
\usepackage{titlesec}  % 用于标题格式

% Set main language to Chinese using xeCJK
\usepackage{xeCJK}

% Fonts
\setmainfont{Times New Roman}
\setCJKmainfont[AutoFakeBold]{SimSun}

% Custom section formatting
\titleformat{\section}{\Large\bfseries\color{blue!70!black}}{}{0em}{}
\titlespacing{\section}{0pt}{12pt}{12pt}

% Define column types for tables to allow wrapping
\newcolumntype{L}[1]{>{\raggedright\arraybackslash}p{#1}}
\newcolumntype{C}[1]{>{\centering\arraybackslash}p{#1}}

\title{\textbf{\huge 人教版二年级上册第一单元\\核心字词精讲手册}}
\author{整理自:NotebookLM 生成资料}
\date{}

\begin{document}

\maketitle

\section{一、 写字表详解}
\textit{重点关注:多音字(加粗标注)与形近字的区别。}

\vspace{0.5cm}

\begin{longtable}{C{1.5cm} C{1.5cm} L{5cm} L{5cm}}
    \toprule
    \textbf{生字} & \textbf{拼音} & \textbf{组词} & \textbf{形近字及组词} \\
    \midrule
    \endhead
    
    \textbf{两} & liǎng & 两个、两旁、斤两 & \textbf{雨}(雨水) \\
    \midrule
    \textbf{就} & jiù & 就是、成就、迁就 & \textbf{京}(北京) \\
    \midrule
    \textbf{哪} & nǎ & 哪里、哪边 \newline \textbf{na} (天哪) \newline \textbf{né} (哪吒) & \textbf{那}(那个) \\
    \midrule
    \textbf{宽} & kuān & 宽广、宽阔、宽容 & \textbf{觉}(觉得) \\
    \midrule
    \textbf{顶} & dǐng & 头顶、山顶、顶层 & \textbf{须}(胡须) \\
    \midrule
    \textbf{肚} & dù & 肚子、肚皮 \newline \textbf{dǔ} (猪肚) & \textbf{杜}(杜鹃) \\
    \midrule
    \textbf{皮} & pí & 皮肤、皮球、书皮 & \textbf{波}(波浪) \\
    \midrule
    \textbf{孩} & hái & 孩子、小孩、女孩 & \textbf{该}(应该) \\
    \midrule
    \textbf{跳} & tiào & 跳高、心跳、跳远 & \textbf{桃}(桃子) \\
    \midrule
    \textbf{变} & biàn & 变化、改变、变成 & \textbf{弯}(弯曲) \\
    \midrule
    \textbf{极} & jí & 北极、极点、好极了 & \textbf{吸}(呼吸) \\
    \midrule
    \textbf{片} & piàn & 一片、叶片、照片 \newline \textbf{piān} (相片儿) & \textbf{版}(出版) \\
    \midrule
    \textbf{傍} & bàng & 傍晚、依傍 & \textbf{旁}(旁边) \\
    \midrule
    \textbf{海} & hǎi & 大海、海水、海洋 & \textbf{梅}(梅花) \\
    \midrule
    \textbf{洋} & yáng & 海洋、洋气、大洋 & \textbf{样}(样子) \\
    \midrule
    \textbf{作} & zuò & 作业、工作、动作 & \textbf{昨}(昨天) \\
    \midrule
    \textbf{坏} & huài & 坏人、坏事、破坏 & \textbf{怀}(怀抱) \\
    \midrule
    \textbf{给} & gěi & 送给、交给 \newline \textbf{jǐ} (给予) & \textbf{绘}(绘画) \\
    \midrule
    \textbf{法} & fǎ & 办法、方法、法律 & \textbf{去}(去年) \\
    \midrule
    \textbf{如} & rú & 如果、犹如、如今 & \textbf{好}(美好) \\
    \midrule
    \textbf{脚} & jiǎo & 脚步、双脚、脚下 & \textbf{腿}(大腿) \\
    \midrule
    \textbf{它} & tā & 它们、其它 & \textbf{安}(安全) \\
    \midrule
    \textbf{娃} & wá & 娃娃、女娃 & \textbf{挂}(挂画) \\
    \midrule
    \textbf{她} & tā & 她们、她的 & \textbf{他}(他们) \\
    \midrule
    \textbf{毛} & máo & 皮毛、毛巾、羽毛 & \textbf{手}(双手) \\
    \midrule
    \textbf{更} & gèng & 更加、更好 \newline \textbf{gēng} (更新) & \textbf{便}(方便) \\
    \midrule
    \textbf{知} & zhī & 知识、知道、知了 & \textbf{和}(和平) \\
    \bottomrule
\end{longtable}

\newpage

\section{二、 词语表详解}
\textit{重点积累:词语的近义词、反义词及造句运用。}

\vspace{0.5cm}

\begin{longtable}{L{2cm} L{1.8cm} L{5.5cm} L{2cm} L{2.2cm}}
    \toprule
    \textbf{词语} & \textbf{拼音} & \textbf{造句} & \textbf{近义词} & \textbf{反义词} \\
    \midrule
    \endhead
    
    \textbf{妈妈} & mā ma & 我最爱我的妈妈。 & 母亲 & 爸爸 \\
    \midrule
    \textbf{身子} & shēn zi & 小猫的身子软绵绵的。 & 身体 & / \\
    \midrule
    \textbf{他们} & tā men & 他们正在操场上踢足球。 & / & 她们/它们 \\
    \midrule
    \textbf{看见} & kàn jiàn & 我看见天空中有一道彩虹。 & 瞧见 & / \\
    \midrule
    \textbf{我们} & wǒ men & 我们是祖国的花朵。 & 咱们 & 你们/他们 \\
    \midrule
    \textbf{哪里} & nǎ lǐ & 请问图书馆在哪里? & 何处 & 这里/那里 \\
    \midrule
    \textbf{那边} & nà biān & 那边的风景真美啊! & 那里 & 这边 \\
    \midrule
    \textbf{雪白} & xuě bái & 天鹅长着一身雪白的羽毛。 & 洁白 & 漆黑/乌黑 \\
    \midrule
    \textbf{过去} & guò qù & 事情已经过去了,就不要再提了。 & 往昔 & 未来/现在 \\
    \midrule
    \textbf{孩子} & hái zi & 所有的孩子都喜欢听故事。 & 儿童 & 大人/长辈 \\
    \midrule
    \textbf{什么} & shén me & 你在画什么? & / & / \\
    \midrule
    \textbf{太阳} & tài yáng & 太阳从东方升起。 & 旭日 & 月亮 \\
    \midrule
    \textbf{天空} & tiān kōng & 蓝蓝的天空飘着几朵白云。 & 苍穹 & 大地 \\
    \midrule
    \textbf{一起} & yī qǐ & 我和好朋友一起去上学。 & 一同 & 单独 \\
    \midrule
    \textbf{冬天} & dōng tiān & 冬天到了,天气变冷了。 & 冬季 & 夏天 \\
    \midrule
    \textbf{花朵} & huā duǒ & 花园里开满了五颜六色的花朵。 & 花儿 & / \\
    \midrule
    \textbf{池子} & chí zi & 那个池子里养了许多金鱼。 & 池塘 & / \\
    \midrule
    \textbf{江河} & jiāng hé & 奔腾的江河汇入大海。 & 河流 & / \\
    \midrule
    \textbf{海洋} & hǎi yáng & 海洋里生活着各种各样的生物。 & 大海 & 陆地 \\
    \midrule
    \textbf{许多} & xǔ duō & 树林里有许多小鸟。 & 很多 & 极少/少量 \\
    \midrule
    \textbf{田地} & tián dì & 农民伯伯在田地里辛勤劳作。 & 田野 & / \\
    \midrule
    \textbf{工作} & gōng zuò & 爸爸工作很认真。 & 劳动 & 休息 \\
    \midrule
    \textbf{办法} & bàn fǎ & 乌鸦想出了喝水的办法。 & 方法 & / \\
    \midrule
    \textbf{你们} & nǐ men & 你们明天去哪里玩? & / & 我们 \\
    \midrule
    \textbf{如果} & rú guǒ & 如果明天下雨,运动会就延期。 & 假如 & / \\
    \midrule
    \textbf{已经} & yǐ jīng & 我已经把作业写完了。 & 已然 & 未曾 \\
    \midrule
    \textbf{长大} & zhǎng dà & 我想快点长大,去探索世界。 & 成长 & / \\
    \midrule
    \textbf{告别} & gào bié & 燕子告别了南方,飞向北方。 & 辞别 & 重逢/相聚 \\
    \midrule
    \textbf{四海为家} & sì hǎi wéi jiā & 蒲公英的孩子四海为家。 & 浪迹天涯 & 安土重迁 \\
    \midrule
    \textbf{自己} & zì jǐ & 我们要学会自己的事情自己做。 & 自身 & 别人/他人 \\
    \midrule
    \textbf{出发} & chū fā & 探险队整装待发,准备出发。 & 启程 & 到达/抵达 \\
    \midrule
    \textbf{动物} & dòng wù & 我们要爱护小动物。 & / & 植物 \\
    \midrule
    \textbf{胆子} & dǎn zi & 他的胆子很小,不敢一个人走夜路。 & 胆量 & / \\
    \midrule
    \textbf{肚子} & dù zi & 我饿得肚子咕咕叫。 & 腹部 & / \\
    \midrule
    \textbf{那里} & nà lǐ & 那里有一片茂密的森林。 & 那边 & 这里 \\
    \midrule
    \textbf{知识} & zhī shi & 书籍是知识的海洋。 & 学问 & 常识 \\
    \bottomrule
\end{longtable}

\end{document}