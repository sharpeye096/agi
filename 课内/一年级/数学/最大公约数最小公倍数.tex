\documentclass[12pt, a4paper]{article}

% --- PREAMBLE ---
\usepackage[a4paper, top=2.5cm, bottom=2.5cm, left=2.5cm, right=2.5cm]{geometry}
\usepackage{fontspec}
\usepackage{amsmath}
\usepackage{amssymb} % Added for \checkmark
\usepackage{graphicx}
\usepackage{tcolorbox} % For highlighted boxes

% Set main language to Chinese
\usepackage[chinese, bidi=basic, provide=*]{babel}
\babelprovide[import, onchar=ids fonts]{chinese}
\babelprovide[import, onchar=ids fonts]{english}

% Fonts - Using Noto Serif for a book-like feel
\babelfont{rm}{Noto Serif}
\babelfont[chinese]{rm}{Noto Serif CJK SC}

% Custom section formatting
\usepackage{titlesec}
\titleformat{\section}{\Large\bfseries\color{blue!70!black}}{}{0em}{}
\titleformat{\subsection}{\large\bfseries}{}{0em}{}
\titlespacing{\section}{0pt}{12pt}{6pt}

% Packages for lists
\usepackage{enumitem}
\setlist[itemize]{label=$\bullet$}

% Command for Short Division display
\newcommand{\shortdiv}[5]{
    \begin{tabular}{r|ll}
    #1 & #2 & #3 \\ \cline{2-3}
    #4 & #5 & 
    \end{tabular}
}

\title{\textbf{数字小侦探:寻找最大公约数和最小公倍数}}
\author{小学数学趣味讲义}
\date{}

\begin{document}

\maketitle

亲爱的小朋友,欢迎来到数字侦探社!今天我们要去破解两个神秘的数字密码,它们的名字有点长,一个叫\textbf{最大公约数},一个叫\textbf{最小公倍数}。

别被名字吓跑了,其实它们非常好玩!

\hrule
\vspace{0.5cm}

\section{第一关:最大公约数(GCD)—— 切蛋糕大师}

\subsection*{1. 什么是“约数”(因数)?}
想象一下,你有一个数字 \textbf{12}。如果我们把它看作一块 \textbf{12厘米长的蛋糕},我们要把它切成整数厘米的小块,而且不能有剩余。我们可以切成多长?

\begin{itemize}
    \item 1厘米(12块) \checkmark
    \item 2厘米(6块) \checkmark
    \item 3厘米(4块) \checkmark
    \item 4厘米(3块) \checkmark
    \item 6厘米(2块) \checkmark
    \item 12厘米(1块) \checkmark
    \item 5厘米?不行,切完还剩一点点渣渣。 $\times$
\end{itemize}

这些能刚好把 12 分完的数字(1, 2, 3, 4, 6, 12),就是 \textbf{12 的约数}。

\subsection*{2. 什么是“最大公约数”?}
现在,我们的桌子上放着两块蛋糕:
\begin{itemize}
    \item 一块长 \textbf{12} 厘米
    \item 一块长 \textbf{18} 厘米
\end{itemize}

\textbf{任务}:要把它们切成\textbf{同样长}的小段,而且每段要\textbf{尽可能长},不能有浪费。

\begin{itemize}
    \item \textbf{12 的约数}:1, 2, 3, 4, \textbf{6}, 12
    \item \textbf{18 的约数}:1, 2, 3, \textbf{6}, 9, 18
\end{itemize}

\textbf{发现了吗?} 它们都有 1, 2, 3, 6。这些叫\textbf{公约数}。其中\textbf{最大}的一个是 \textbf{6}。

\begin{tcolorbox}[colback=yellow!10!white, colframe=orange!75!black, title=结论]
\textbf{最大公约数}就是能同时整除两个数,且最大的那个数。
\end{tcolorbox}

\newpage

\section{第二关:最小公倍数(LCM)—— 跳格子比赛}

\subsection*{1. 什么是“倍数”?}
这就简单啦!就像背乘法口诀。
\begin{itemize}
    \item \textbf{3 的倍数}:3, 6, 9, 12, 15... (像小青蛙每次跳 3 格)
    \item \textbf{4 的倍数}:4, 8, 12, 16, 20... (像小兔子每次跳 4 格)
\end{itemize}

\subsection*{2. 什么是“最小公倍数”?}
现在,小青蛙(跳3格)和小兔子(跳4格)在同一起跑线上开始跳。它们会在哪个格子上\textbf{第一次相遇}呢?

\begin{itemize}
    \item \textbf{3 的脚印}:3, 6, 9, \textbf{12}, 15, 18, 21, \textbf{24}...
    \item \textbf{4 的脚印}:4, 8, \textbf{12}, 16, 20, \textbf{24}...
\end{itemize}

\textbf{发现了吗?} 它们在第 12 格、第 24 格都会相遇。这些叫\textbf{公倍数}。其中\textbf{最近(最小)}的一个是 \textbf{12}。

\begin{tcolorbox}[colback=green!10!white, colframe=green!75!black, title=结论]
\textbf{最小公倍数}就是能同时被两个数整除,且最小的那个数。
\end{tcolorbox}

\section{第三关:超级武器——短除法}

每次都列举数字太麻烦啦!侦探们有一个万能公式,叫\textbf{“短除法”}。

\textbf{例子:找出 12 和 18 的最大公约数和最小公倍数}

\vspace{0.5cm}

\begin{center}
\begin{tabular}{r|ll}
 \textbf{2} & 12 & 18 \\ \cline{2-3}
 \textbf{3} & \phantom{1}6 & \phantom{1}9 \\ \cline{2-3}
            & \phantom{1}2 & \phantom{1}3
\end{tabular}
\end{center}

\vspace{0.3cm}
\textbf{步骤解析:}
\begin{enumerate}
    \item 先用能同时整除它们的 \textbf{2} 去除,得到 6 和 9。
    \item 再用能同时整除 6 和 9 的 \textbf{3} 去除,得到 2 和 3。
    \item 2 和 3 互质(除了1没法再除了),结束!
\end{enumerate}

\begin{tcolorbox}[colback=red!5!white, colframe=red!75!black, title=\textbf{终极秘籍(背下来!)}]
\textbf{1. 求最大公约数:只乘左边!} \\
把左边的数字乘起来:$2 \times 3 = \mathbf{6}$ \\
\textit{所以,12 和 18 的最大公约数是 6。}

\vspace{0.2cm}
\textbf{2. 求最小公倍数:左边下面一网打尽!} \\
把左边的数字和下面的数字全部乘起来:$2 \times 3 \times 2 \times 3 = \mathbf{36}$ \\
\textit{所以,12 和 18 的最小公倍数是 36。}
\end{tcolorbox}

\newpage

\section{生活中的数学(为什么要学这个?)}

\begin{enumerate}
    \item \textbf{最大公约数帮你铺地砖}:
    你有一个长方形的房间(比如长 12 米,宽 18 米),你想铺那种\textbf{正方形}的大地砖,又不想要切碎地砖,选多大边长的砖最合适?(答案是 6 米!)
    
    \item \textbf{最小公倍数帮你等公交}:
    一路公交车每 10 分钟来一趟,二路公交车每 15 分钟来一趟。如果你错过了它们同时发车,下一次它们\textbf{同时到达}要等多久?(求 10 和 15 的最小公倍数,答案是 30 分钟!)
\end{enumerate}

\vspace{1cm}

\section{小侦探练兵场}
\textbf{请用“短除法”计算下面每组数的最大公约数和最小公倍数:}

\vspace{0.5cm}

\begin{enumerate}[itemsep=4em]
    \item \textbf{8 和 12}
    \begin{itemize}
        \item 最大公约数:\underline{\hspace{3cm}}
        \item 最小公倍数:\underline{\hspace{3cm}}
    \end{itemize}

    \item \textbf{10 和 20}
    \begin{itemize}
        \item 最大公约数:\underline{\hspace{3cm}}
        \item 最小公倍数:\underline{\hspace{3cm}}
    \end{itemize}

    \item \textbf{15 和 25}
    \begin{itemize}
        \item 最大公约数:\underline{\hspace{3cm}}
        \item 最小公倍数:\underline{\hspace{3cm}}
    \end{itemize}
\end{enumerate}

\newpage

\section{终极挑战:我是解题小能手}
\textbf{这里有两道生活中的难题,请你根据学到的知识,判断是求“最大公约数”还是“最小公倍数”,并算出答案。}

\vspace{0.5cm}

\begin{enumerate}
    \item \textbf{剪彩带(不能浪费哦!)} \\
    老师有两根彩带,红色的长 \textbf{24} 厘米,蓝色的长 \textbf{36} 厘米。老师想把它们剪成\textbf{同样长}的小段,而且不能有剩余。每段彩带\textbf{最长}可以是几厘米?
    \vspace{4cm}

    \item \textbf{猜猜有多少糖果?} \\
    桌子上有一袋糖果,如果平均分给 \textbf{6} 个小朋友,正好分完;如果平均分给 \textbf{8} 个小朋友,也正好分完。这袋糖果\textbf{至少}有多少颗?
    \vspace{4cm}
\end{enumerate}

\end{document}