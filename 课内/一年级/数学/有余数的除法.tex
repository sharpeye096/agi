\documentclass[12pt, a4paper]{article}

% --- PREAMBLE ---
\usepackage[a4paper, top=2.5cm, bottom=2.5cm, left=2.5cm, right=2.5cm]{geometry}
\usepackage{fontspec}
\usepackage{amsmath}
\usepackage{amssymb}
\usepackage{graphicx}
\usepackage{tcolorbox}
\usepackage{array} % For better table alignment

% Set main language to Chinese
\usepackage[chinese, bidi=basic, provide=*]{babel}
\babelprovide[import, onchar=ids fonts]{chinese}
\babelprovide[import, onchar=ids fonts]{english}

% Fonts
\babelfont{rm}{Noto Serif}
\babelfont[chinese]{rm}{Noto Serif CJK SC}

% Custom section formatting
\usepackage{titlesec}
\titleformat{\section}{\Large\bfseries\color{purple!70!black}}{}{0em}{}
\titleformat{\subsection}{\large\bfseries}{}{0em}{}
\titlespacing{\section}{0pt}{12pt}{6pt}

% Packages for lists
\usepackage{enumitem}
\setlist[itemize]{label=$\bullet$}

\title{\textbf{数学王国奇遇记:有余数的除法}}
\author{小学数学趣味讲义}
\date{}

\begin{document}

\maketitle

亲爱的小朋友,欢迎来到数学王国的“浆果乐园”!今天我们要解决一个关于\textbf{分享}的问题。当你想要平均分东西,却发现\textbf{分不完}的时候,该怎么办呢?

秘密就藏在今天的课里——\textbf{有余数的除法}。

\hrule
\vspace{0.5cm}

\section{第一部分:分草莓的故事(理解意义)}

\subsection*{1. 遇到的问题}
熊大摘了 \textbf{14} 颗又大又红的草莓,他想把这些草莓\textbf{平均分}给 \textbf{4} 个好朋友。
请问:\textbf{每个朋友能分到几颗?还剩几颗?}

\begin{tcolorbox}[colback=pink!10!white, colframe=purple!75!black, title=动动手想一想]
    我们可以用圆圈代表草莓,画一画:
    
    \vspace{0.2cm}
    $\bigcirc \bigcirc \bigcirc \bigcirc$ (分给第1个朋友)\\
    $\bigcirc \bigcirc \bigcirc \bigcirc$ (分给第2个朋友)\\
    $\bigcirc \bigcirc \bigcirc \bigcirc$ (分给第3个朋友)\\
    $\bigcirc \bigcirc$ (哎呀,只剩2个了,不够分给第4个朋友了!)
\end{tcolorbox}

\subsection*{2. 用数学语言说话}
这个过程,我们可以用一个除法算式来表示:

\[ 14 \div 4 = 3 \cdots\cdots 2 \]

\begin{itemize}
    \item \textbf{14} 是\textbf{被除数}(一共有14颗草莓)。
    \item \textbf{4} 是\textbf{除数}(平均分给4个人)。
    \item \textbf{3} 是\textbf{商}(每人分到3颗)。
    \item \textbf{2} 是\textbf{余数}(最后剩下2颗,没法分了)。
    \item $\cdots\cdots$ 是\textbf{省略号},读作“余”。
\end{itemize}

\textbf{读法}:14除以4等于3余2。

\newpage

\section{第二部分:竖式计算四步走(掌握方法)}

怎样用竖式(像盖房子一样)把这个算式算出来呢?请记住这句口诀:
\textbf{“一商、二乘、三减、四比”}。

\vspace{0.5cm}

\begin{center}
\renewcommand{\arraystretch}{1.5}
\begin{tabular}{r@{\quad}l}
    \textbf{步骤图解} & \textbf{意思解释} \\
    \hline
    \begin{tabular}{r@{\hspace{2pt}}l}
         & 3 \\[-3pt] % 商
      \cline{2-2}
      4 \big) & 14 \\
    \end{tabular} 
    & 
    \parbox{8cm}{\textbf{1. 一商}:\\ 想口诀“四(三)十二”,最接近14且比14小。\\ 商是3,写在个位上面。} \\[1.5cm]
    
    \begin{tabular}{r@{\hspace{2pt}}l}
         & 3 \\[-3pt]
      \cline{2-2}
      4 \big) & 14 \\
         & 12 \\[-3pt]
    \end{tabular} 
    & 
    \parbox{8cm}{\textbf{2. 二乘}:\\ 用除数4去乘商3。\\ $4 \times 3 = 12$。\\ 把12写在被除数14的下面。意思是\textbf{分掉了12个}。} \\[1.5cm]
    
    \begin{tabular}{r@{\hspace{2pt}}l}
         & 3 \\[-3pt]
      \cline{2-2}
      4 \big) & 14 \\
         & 12 \\[-3pt]
      \cline{2-2}
         &  2
    \end{tabular} 
    & 
    \parbox{8cm}{\textbf{3. 三减}:\\ 用被除数减去分掉的数。\\ $14 - 12 = 2$。\\ 2就是\textbf{余数}(剩下的)。} \\[1.5cm]
\end{tabular}
\end{center}

\begin{tcolorbox}[colback=yellow!10!white, colframe=orange!80!black, title=\textbf{4. 四比(最重要的一步!)}]
    最后,一定要看一眼余数!
    
    \textbf{余数必须比除数小!} ($2 < 4$)
    
    \textit{如果余数比除数大,说明刚才商太小了,还能再分!}
\end{tcolorbox}

\newpage

\section{第三部分:小试牛刀(练习时间)}

\textbf{1. 圈一圈,填一填。}
\vspace{0.5cm}

有 11 个苹果 $\bigcirc \bigcirc \bigcirc \bigcirc \bigcirc \bigcirc \bigcirc \bigcirc \bigcirc \bigcirc \bigcirc$
\\
每 3 个装一盘,可以装( \hspace{1cm} )盘,还剩( \hspace{1cm} )个。

\[ 11 \div 3 = \underline{\hspace{1cm}} \cdots\cdots \underline{\hspace{1cm}} \]

\vspace{0.5cm}
\textbf{2. 用竖式计算下面各题。}
\vspace{0.5cm}

\begin{tabular}{p{4cm} p{4cm} p{4cm}}
    (1) $20 \div 6 =$ & (2) $34 \div 5 =$ & (3) $45 \div 7 =$ \\
    & & \\
    & & \\
    \hspace{0.5cm}
    \begin{tabular}{r@{\hspace{2pt}}l}
         & \phantom{0} \\[-3pt]
      \cline{2-2}
      6 \big) & 20 \\
         & \phantom{0} \\[-3pt]
      \cline{2-2}
         & \phantom{0}
    \end{tabular}
    &
    \hspace{0.5cm}
    \begin{tabular}{r@{\hspace{2pt}}l}
         & \phantom{0} \\[-3pt]
      \cline{2-2}
      5 \big) & 34 \\
         & \phantom{0} \\[-3pt]
      \cline{2-2}
         & \phantom{0}
    \end{tabular}
    &
    \hspace{0.5cm}
    \begin{tabular}{r@{\hspace{2pt}}l}
         & \phantom{0} \\[-3pt]
      \cline{2-2}
      7 \big) & 45 \\
         & \phantom{0} \\[-3pt]
      \cline{2-2}
         & \phantom{0}
    \end{tabular}
    \\[2cm] % Add extra space for calculation
\end{tabular}

\section{第四部分:思维挑战(生活中的余数)}

\begin{enumerate}
    \item \textbf{坐船问题}:\\
    有 26 个小朋友去划船,每条船最多坐 4 人。至少要租几条船?
    \vspace{0.5cm}
    
    算式:\underline{\hspace{5cm}}
    
    \textit{想一想:剩下的 2 个小朋友也要坐船吗?答案是 \underline{\hspace{2cm}} 条。}
    \vspace{1cm}

    \item \textbf{排队问题}:\\
    同学们按“红、黄、蓝”的顺序挂气球。第 16 个气球是什么颜色?
    \vspace{0.5cm}
    
    算式:\underline{\hspace{5cm}}
    
    \textit{想一想:看余数是几,就是第几个颜色。}
\end{enumerate}

\end{document}