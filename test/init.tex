\documentclass[12pt, a4paper]{article}

% --- PREAMBLE ---
\usepackage[a4paper, top=2cm, bottom=2cm, left=2.5cm, right=2.5cm]{geometry}
\usepackage{fontspec}
\usepackage{xcolor}
\usepackage{tcolorbox} % For fun fact boxes
\usepackage{graphicx}
\usepackage{setspace}

% Set main language to Chinese using xeCJK (better for xelatex on Windows)
\usepackage{xeCJK}
\setCJKmainfont{SimSun} % Standard Windows font, or use "Microsoft YaHei"
\setmainfont{Times New Roman} % Standard English font


% Custom commands
\newcommand{\funfact}[1]{
    \begin{tcolorbox}[colback=yellow!10!white, colframe=orange!75!black, title=\textbf{💡 小知识}]
        #1
    \end{tcolorbox}
}

\newcommand{\imageplaceholder}[2]{
    \par
    \vspace{5cm} % 纯空白区域,无边框
    \par
}

\title{\textbf{\huge 地球的忙碌旅行}}
\author{给二年级的科普小故事}
\date{}

\begin{document}

\maketitle
\large
\setstretch{1.5}

你好呀,小小探险家!🌍

你知道吗?我们脚下的地球,其实是一个非常忙碌的“旅行家”。它每天都在不停地动,而且它有两种不同的动法。正是因为这两种运动,我们才有了白天、黑夜,还有春夏秋冬。

让我们一起来看看地球是怎么忙碌的吧!

\section{第一关:为什么会有白天和黑夜?}

\subsection*{地球是个“大陀螺” (自转)}

想象一下,地球就像一个在桌子上不停旋转的**大陀螺**。地球转呀转,从来不休息。这种自己转圈圈的运动,我们叫它——\textbf{自转}。

\begin{itemize}
    \item 地球转一圈,需要多长时间?\textbf{整整 24 个小时(一天)}。
\end{itemize}

\imageplaceholder{自转示意图}{建议图片:地球像一个可爱的卡通陀螺在旋转,一半亮一半暗。}

\subsection*{太阳是个“大灯泡”}

现在,想象太阳是一个一直亮着的**超级大灯泡**。

当地球这个“大陀螺”转动的时候:
\begin{itemize}
    \item 面对太阳的那一面,被照亮了,这就是\textbf{白天}(我们要上学、玩耍)。
    \item 背对太阳的那一面,光照不到,黑乎乎的,这就是\textbf{黑夜}(我们要睡觉啦)。
\end{itemize}

因为地球不停地转,所以我们一会儿在亮的一面,一会儿在暗的一面,白天和黑夜就这样轮流出现啦!

\funfact{虽然我们感觉不到地球在动,但它转得可快了!就像坐在平稳的高铁上,你感觉不到车在跑,但窗外的风景(太阳和星星)却在“后退”。}

\newpage

\section{第二关:为什么会有春夏秋冬?}

\subsection*{围着篝火跑圈圈 (公转)}

地球不仅自己转(自转),它还很喜欢围着太阳跑。

地球沿着一条大大的跑道,围着太阳转大圈圈。这种绕着太阳跑的运动,我们叫它——\textbf{公转}。

\begin{itemize}
    \item 地球跑完这一大圈,需要多长时间?\textbf{365 天(一年)}。
\end{itemize}

\imageplaceholder{公转示意图}{建议图片:太阳在中间,地球沿着一条椭圆形的轨道跑,跑道上标记着春夏秋冬。}

\subsection*{关键秘密:地球是“歪”着身子的!}

这才是春夏秋冬的真正秘密!

地球在跑圈的时候,不是直挺挺地站着的,而是\textbf{歪着脑袋、斜着身子}跑的(就像你侧着身子跑步一样)。

这就导致了一个神奇的现象:
\begin{itemize}
    \item \textbf{夏天的时候}:地球的上半身(北半球,就是我们住的地方)\textbf{侧向太阳}。太阳光直直地照在我们头顶,就像你把脸凑近篝火,感觉非常热!🔥
    \item \textbf{冬天的时候}:地球跑到了跑道的另一边,这时候它的上半身是\textbf{背向太阳}歪的。太阳光只能斜斜地照过来,就像你离篝火远了一点,感觉就冷了!❄️
    \item \textbf{春天和秋天}:地球侧着身子,不远也不近,所以不冷也不热,最舒服!🍂🌸
\end{itemize}

\imageplaceholder{倾斜的地球}{建议图片:地球歪着轴,一边是夏天(阳光直射),一边是冬天(阳光斜射)。}

\funfact{当我们在过炎热的夏天(吃西瓜)时,住在地球另一端(南半球,比如澳大利亚)的小朋友正在过冬天(堆雪人)呢!因为地球歪着身子,把屁股对着太阳啦!}

\newpage

\section*{附录:图片生成提示词 (Image Prompts)}

爸爸妈妈可以使用这些“咒语”,用 AI 画出辅助理解的图片哦:

\textbf{1. 解释白天黑夜(自转):}
\begin{quote}
    \textit{Cute 3D cartoon earth character spinning like a top in space, half of the earth is illuminated by a bright sun lamp nearby, the other half is in shadow with sleeping zzz symbols, educational illustration style, bright colors, high quality.}
    \\ (中文意图:可爱的3D卡通地球像陀螺一样旋转,一半被太阳灯照亮,另一半在阴影里睡觉。)
\end{quote}

\textbf{2. 解释公转(绕太阳跑):}
\begin{quote}
    \textit{Solar system diagram for kids, a cute smiling sun in the center, earth running on a circular track around the sun, dashed lines showing the orbit, flat vector illustration, simple and colorful.}
    \\ (中文意图:儿童版太阳系图,中间是笑脸太阳,地球沿着轨道跑,扁平矢量风格。)
\end{quote}

\textbf{3. 解释季节(倾斜的地球):}
\begin{quote}
    \textit{Close up of cartoon earth tilting on its axis, one side leaning towards a warm sun (summer scene with sunglasses), the other side leaning away (winter scene with scarf), split comparison, educational infographic style.}
    \\ (中文意图:倾斜的地球特写,一边倾向太阳是夏天戴墨镜,一边远离太阳是冬天戴围巾。)
\end{quote}

\end{document}